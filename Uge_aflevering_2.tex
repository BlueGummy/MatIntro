\documentclass[a4paper]{article}
\usepackage{amsmath} % flere matematikkommandoer
\usepackage[utf8]{inputenc} % æøå
\usepackage[T1]{fontenc} % mere æøå
\usepackage[Danish]{babel} % orddeling
\usepackage{verbatim} % så man kan skrive ren tekst
\usepackage[all]{xy} % den sidste (avancerede) formel i dokumentet
\usepackage{amsmath}
\usepackage{amsfonts}
\usepackage{amssymb}
\usepackage{graphicx}
\usepackage{fancyhdr}
\usepackage{moreverb}
\usepackage{indentfirst}
\usepackage{graphicx}
\graphicspath{ {images/} }



\title{Uge aflevering 1}
\author{Carsten Ejstrup, Jesper Henrichsen, Thomas Broby Nielsen}

\begin{document}
\maketitle

\tableofcontents
\pagebreak
\section{Opgave 2}
\indent A) Find den fuldstændige løsning til differantialligningen. 
$y'' + 2y' - 3y = 0$
\\
\\
Til at finde den fuldstændige løsning, benytter vi os af den karaktistiske ligning :
$r^2 + pr + q = 0$
som når man fylder tallene ind fra den ligning vi er blevet givet, får vi: 
$r^2 + 2r - 3 = 0$
Da dette er et karaktaristik andengrads polynomium, begynder vi med at finde røderne $r_1$ og $r_2$. Formlen for andengrads polynomiet er givet ved:
$$r = \frac{-b \pm \sqrt[]{b^2-4ac}}{2a}$$
herfra får vi:
$$r = \frac{-2 \pm \sqrt[]{2^2-4\cdot 1 \cdot (-3)}}{2\cdot 1} = \frac{-2 \pm \sqrt[]{16}}{2}$$
Fra dette får vi $r_1$ og $r_2$ til at være lig: 
$$
\frac{-2 \pm 3}{2}\left\}\begin{array}{l}
r_1 = 1 \\
r_2 = -3
\end{array}\right.
$$
Derved er den fuldstændige løsning givet ved $y = Ce^{r_1x} + De^{r_2x}$\footnote{Se sætning 10.5.3, på side 531 i TLO}, som når vi indsætter vores variabler bliver til $y = Ce^{x}+De^{-3x}$. Og vi kan derved finde vores y' til at være $y = Ce^{x}-3De^{-3x}$


\end{document}